\documentclass[fleqn]{article}
\usepackage[margin=1in]{geometry}
\usepackage[nodisplayskipstretch]{setspace}
\usepackage{amsmath, nccmath, bm}
\usepackage{amssymb}
\usepackage{enumitem}
\usepackage{graphicx}
\usepackage{float}
\usepackage{listings}
\usepackage{hyperref}
\usepackage[svgnames]{xcolor}
\graphicspath{{./images}}

\hypersetup{
    colorlinks=true,
    linkcolor=black,
    filecolor=black,      
    urlcolor=blue
    }

\newcommand{\zerodisplayskip}{
	\setlength{\abovedisplayskip}{0pt}%
	\setlength{\belowdisplayskip}{0pt}%
	\setlength{\abovedisplayshortskip}{0pt}%
	\setlength{\belowdisplayshortskip}{0pt}%
	\setlength{\mathindent}{0pt}}
	
\definecolor{vgreen}{RGB}{104,180,104}
\definecolor{vblue}{RGB}{49,49,255}
\definecolor{vorange}{RGB}{255,143,102}

\lstdefinestyle{verilog-style}
{
    language=Verilog,
    basicstyle=\small\ttfamily,
    keywordstyle=\color{vblue},
    identifierstyle=\color{black},
    commentstyle=\color{vgreen},
    numbers=left,
    numberstyle=\tiny\color{black},
    numbersep=10pt,
    tabsize=8,
    moredelim=*[s][\colorIndex]{[}{]},
    literate=*{:}{:}1
}

\lstset{style={verilog-style},showstringspaces=false}

\makeatletter
\newcommand*\@lbracket{[}
\newcommand*\@rbracket{]}
\newcommand*\@colon{:}
\newcommand*\colorIndex{%
    \edef\@temp{\the\lst@token}%
    \ifx\@temp\@lbracket \color{black}%
    \else\ifx\@temp\@rbracket \color{black}%
    \else\ifx\@temp\@colon \color{black}%
    \else \color{vorange}%
    \fi\fi\fi
}
\makeatother

\newcommand{\code}[1]{%
	\colorbox{Gainsboro}{\texttt{#1}}%
}

\title{Lab 1}
\author{Owen Sowatzke}
\date{March 17, 2025}

\begin{document}

	\offinterlineskip
	\setlength{\lineskip}{12pt}
	\zerodisplayskip
	\maketitle
	
	\section{Introduction}
	
	\section{Procedure}
	
	\section{Results}
	
	\subsection{NAND Gate}
	
	\subsubsection{Design}
	
	The pull-down circuit of the NAND gate is composed of two sequential NMOS gates. The pull-up circuit is the complement and is composed of two parallel PMOS gates. Note that we also increase the width of the PMOS transistor by a factor of 2 to account for the lower mobility of holes in the p-type material. The schematic for the resulting circuit is shown in Figure \ref{fig::nand_schematic}.
	
	\begin{figure}[H]
		\centerline{\includegraphics[width=0.4\textwidth]{nand_schematic.png}}
		\caption{NAND Circuit Schematic}
		\label{fig::nand_schematic}
	\end{figure}

	We also create a circuit symbol for the NAND gate, which is shown in Figure \ref{fig::nand_symbol}.
	
	\begin{figure}[H]
		\centerline{\includegraphics[width=0.4\textwidth]{nand_symbol.png}}
		\caption{NAND Circuit Symbol}
		\label{fig::nand_symbol}
	\end{figure}
	
	\begin{figure}[H]
		\centerline{\includegraphics[width=0.4\textwidth]{nor_schematic.png}}
		\caption{NOR Circuit Schematic}
		\label{fig::nor_schematic}
	\end{figure}
	
	\begin{figure}[H]
		\centerline{\includegraphics[width=0.4\textwidth]{nor_symbol.png}}
		\caption{NOR Circuit Symbol}
		\label{fig::nor_symbol}
	\end{figure}
	
	\subsection{NAND Gate}

\end{document}