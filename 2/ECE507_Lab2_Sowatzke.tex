\documentclass{article}
\usepackage[margin=1in]{geometry}
\usepackage[nodisplayskipstretch]{setspace}
\usepackage{amsmath, nccmath, bm}
\usepackage{amssymb}
\usepackage{enumitem}
\usepackage{graphicx}
\usepackage{float}
\usepackage{listings}
\usepackage{hyperref}
\usepackage[svgnames]{xcolor}
\usepackage{indentfirst}
%\usepackage{chngcntr}
%\counterwithin{table}{section}
\graphicspath{
{./images}
{./images/nand}
{./images/nor}}

%\hypersetup{
%    colorlinks=true,
%    linkcolor=black,
%    filecolor=black,      
%    urlcolor=blue
%    }

\newcommand{\zerodisplayskip}{
	\setlength{\abovedisplayskip}{0pt}%
	\setlength{\belowdisplayskip}{0pt}%
	\setlength{\abovedisplayshortskip}{0pt}%
	\setlength{\belowdisplayshortskip}{0pt}%
	\setlength{\mathindent}{0pt}}
	
\definecolor{vgreen}{RGB}{104,180,104}
\definecolor{vblue}{RGB}{49,49,255}
\definecolor{vorange}{RGB}{255,143,102}

\lstdefinestyle{verilog-style}
{
    language=Verilog,
    basicstyle=\small\ttfamily,
    keywordstyle=\color{vblue},
    identifierstyle=\color{black},
    commentstyle=\color{vgreen},
    numbers=left,
    numberstyle=\tiny\color{black},
    numbersep=10pt,
    tabsize=8,
    moredelim=*[s][\colorIndex]{[}{]},
    literate=*{:}{:}1
}

\lstset{style={verilog-style},showstringspaces=false}

\lstdefinestyle{nocoloring}{
    keywordstyle=\color{black},
    commentstyle=\color{black},
    stringstyle=\color{black}
}

\makeatletter
\newcommand*\@lbracket{[}
\newcommand*\@rbracket{]}
\newcommand*\@colon{:}
\newcommand*\colorIndex{%
    \edef\@temp{\the\lst@token}%
    \ifx\@temp\@lbracket \color{black}%
    \else\ifx\@temp\@rbracket \color{black}%
    \else\ifx\@temp\@colon \color{black}%
    \else \color{vorange}%
    \fi\fi\fi
}
\makeatother

\newcommand{\code}[1]{%
	\colorbox{Gainsboro}{\texttt{#1}}%
}

\title{Lab 2}
\author{Owen Sowatzke}
\date{March 31, 2025}

\begin{document}

	% \offinterlineskip
	% \setlength{\lineskip}{12pt}
	% \zerodisplayskip
	\maketitle
	
	\section{Introduction}
	
	In this lab, we learn how to create transistor-level layouts of NAND and NOR gates using MAGIC. After creating our layouts, we verify them using DRC (design rule checks). Then, we extract SPICE netlists for simulation. Finally, we simulate these netlists with ngspice and compare our results to schematic simulations. 
	
	\section{NAND Gate}
	
	In this section, we create the layout of a NAND gate and examine its contents. Then, we validate our layout by performing DRC. After our design is validated, we extract a SPICE netlist. We also create a comparable schematic. Finally, we compare our layout-based netlists to the schematic in simulation. In these simulations, we specifically analyze the voltage transfer characteristics (VTC), noise margins, propagation delays, and power consumption of our gate. Comparable results confirm the correctness of our layout.
	
	\subsection{Layout}
	
	In this section, we review our NAND gate layout, which is displayed in Figure \ref{fig::nand_layout}. The NAND gate is composed of two NMOS transistors in series and two PMOS transistors in parallel. The NMOS transistors lie along the n-diffusion region and the PMOS transistor lie along the p-diffusion region. The n-diffusion region is placed in the p-type body, and the p-diffusion region is placed in an n-well. Polysilicon is used for the transistor gates. The layout also includes metal1 wires for ground and VDD. The ground wire is connected to the source of the leftmost NMOS transistor with local interconnect (li) material, and the VDD wire is connected to the source of both PMOS transistors with li material. The drain of the rightmost NMOS transistor is connected to the drain of both PMOS transistors with li material to form the gate output. Metal1 contacts are placed on the gate inputs and outputs to allow for longer-distance routing. Depending on the distance that the I/O must be routed, li material may be sufficient for the connections. To prevent latchup, the ground wire is also connected to p-substrate diffusion with a substrate tap, and the VDD wire is connected to n-substrate diffusion with a well tap.
	
	\begin{figure}[H]
		\centerline{\includegraphics[width=0.3\textwidth]{nand_layout.png}}
		\caption{NAND Gate Layout}
		\label{fig::nand_layout}
	\end{figure}

	Most of the transistor properties are defined by the size of the NMOS and PMOS transistor channels, whose dimensions are shown in Figures \ref{fig::nand_nmos_channel_sizing} and \ref{fig::nand_pmos_channel_sizing}. Note that we have designed the transistors with the minimum channel length for the sky130A technology (150 nm). The channel widths have been chosen to match the delay of an inverter with $1 {\mu}m$ NMOS channel widths and $2 {\mu}m$ PMOS channel widths. With respect to the inverter, the NAND gate NMOS channel widths have been doubled to $2 {\mu}m$ to maintain the same equivalent resistance. The NAND gate PMOS channel widths are kept the same to maintain the same equivalent resistance for the slowest case (when only when transistor is on).
	
	\begin{figure}[H]
		\centerline{\includegraphics[width=0.5\textwidth]{nand_nmos_channel_sizing.png}}
		\caption{Checking NMOS Channel Size}
		\label{fig::nand_nmos_channel_sizing}
	\end{figure}
	
	\begin{figure}[H]
		\centerline{\includegraphics[width=0.5\textwidth]{nand_pmos_channel_sizing.png}}
		\caption{Checking PMOS Channel Size}
		\label{fig::nand_pmos_channel_sizing}
	\end{figure}
	
	For our layout to be valid, it must pass design rule checks (DRC). In Figures \ref{fig::nand_drc_errors_terminal} and \ref{fig::nand_drc_errors_drcmgr}, we confirm that our design meets DRC in two different ways. First, in Figure \ref{fig::nand_drc_errors_terminal}, we use terminal commands to perform a DRC. The \texttt{select} command, selects the entire design, and the \texttt{drc count} and \texttt{drc why} commands return the count and reason for DRC errors respectively. Examining the command outputs, we see that there are no DRC errors. Next, in Figure \ref{fig::nand_drc_errors_drcmgr}, we use the drcmgr to check for DRC errors. Examining the drcmgr output, we see that there are no DRC errors. The primary GUI toolbar also displays a DRC count of 0, which further confirms the validity of our layout.
	
	\begin{figure}[H]
		\centerline{\includegraphics[width=0.5\textwidth]{nand_drc_errors_terminal.png}}
		\caption{Checking DRC Errors from the Terminal}
		\label{fig::nand_drc_errors_terminal}
	\end{figure}
	
	\begin{figure}[H]
		\centerline{\includegraphics[width=0.5\textwidth]{nand_drc_errors_drcmgr.png}}
		\caption{Checking DRC Errors with the drcmgr}
		\label{fig::nand_drc_errors_drcmgr}
	\end{figure}
	
	\subsection{Netlist}
	
	In this section, we generate a netlist from our NAND gate layout. To do so, we specifically follow the commands listed in Figure \ref{fig::nand_netlist_creation}. Note in addition to the commands provided in the lab description we also run the \texttt{ext2spice} command with the \texttt{lvs} option. This option configures the \texttt{ext2spice} settings for LVS (layout vs schematic) simulations \cite{a2021_magic83}.
	
	\begin{figure}[H]
		\centerline{\includegraphics[width=0.3\textwidth]{nand_netlist_creation.png}}
		\caption{SPICE Netlist Extraction}
		\label{fig::nand_netlist_creation}
	\end{figure}
	
	Our NAND gate netlist generated with these commands is displayed in Figure \ref{fig::nand_netlist}. We can validate our transistor connections by examining the contents of the netlist. Doing so, we find that the NMOS transistors \texttt{X1} and \texttt{X2} are connected in series from \texttt{GND} to the output node \texttt{Y}. We also find that their gates are connected to inputs \texttt{B} and \texttt{A} respectively. Next, we find that the PMOS transistors \texttt{X0} and \texttt{X3} are connected in parallel from \texttt{VDD} to the output node \texttt{Y}. Similarly, we find that their gates are connected to inputs \texttt{B} and \texttt{A} respectively.
	
	\begin{figure}[H]
		\lstinputlisting[style=nocoloring,frame=single]{./src/nand.spice}
		\caption{Extracted SPICE Netlist for NAND Gate}
		\label{fig::nand_netlist}
	\end{figure}
	
	\subsection{Schematic Simulation}
	
	In this section, we create a schematic for the NAND gate in xschem. We use this schematic to perform schematic simulations that we can compare to our netlist simulations. We can then use our comparison results to validate the functionality of our layout. Our NAND gate schematic is shown in Figure \ref{fig::nand_schematic}. Note that the channel width and length match those configured in our layout. Additionally, the sequential gates have the same order as the layout (input B controls the gate of the NMOS transistor closest to ground).
	
	\begin{figure}[H]
		\centerline{\includegraphics[width=0.5\textwidth]{nand_schematic.png}}
		\caption{NAND Schematic}
		\label{fig::nand_schematic}
	\end{figure}
	
	\noindent We also create a schematic symbol for our NAND gate, which is shown in Figure \ref{fig::nand_symbol}. This enables us to efficiently reuse our NAND gate in multiple test circuits.
	
	\begin{figure}[H]
		\centerline{\includegraphics[width=0.3\textwidth]{nand_symbol.png}}
		\caption{NAND Schematic Symbol}
		\label{fig::nand_symbol}
	\end{figure}
	
	\subsection{Simulation Results}
	
	In this section, we compare the performance of our layout to the equivalent schematic. We use this comparison to validate the functionality of our layout. We specifically compare the behavior of the layout to the behavior of the schematic by performing VTC analysis, noise analysis, delay analysis, and power analysis.
	
	\subsubsection{VTC}
	
	First, we compare the VTC from the netlist simulations to the VTC from the schematic simulations. To generate a VTC from our netlist, we create a test circuit in SPICE, which is shown in Figure \ref{fig::nand_vtc_test_circuit}. This circuit connects the \texttt{A} input of the circuit to a voltage source which is swept in 1mV steps from 0V to 1.8V, while the \texttt{B} input is connected to \texttt{VDD}.
	
	\begin{figure}[H]
		\lstinputlisting[style=nocoloring,frame=single,basicstyle=\fontsize{7}{7}\selectfont\ttfamily]{./src/nand_vtc.spice}
		\caption{SPICE Test Circuit to Extract VTC from Netlist}
		\label{fig::nand_vtc_test_circuit}
	\end{figure}
	
	The resulting VTC is shown in Figure \ref{fig::nand_vtc}. Looking at the output waveform, we see that the output voltage goes to zero, when both inputs are 1, which is the correct behavior for a NAND gate. Examining the terminal output, we also find that $V_m$, the voltage at which $V_{in} = V_{out}$, is 0.8257V.
	
	\begin{figure}[H]
		\centerline{\includegraphics[width=0.8\textwidth]{nand_vtc.png}}
		\caption{VTC from Netlist Simulation}
		\label{fig::nand_vtc}
	\end{figure}
	
	\noindent We can perform similar analysis for our schematic, using the test circuit shown in Figure \ref{fig::nand_vtc_schem_test_circuit}.
	
	\begin{figure}[H]
		\centerline{\includegraphics[width=0.8\textwidth]{nand_vtc_test_circuit.png}}
		\caption{Schematic Test Circuit to Collect NAND Gate VTC}
		\label{fig::nand_vtc_schem_test_circuit}
	\end{figure}
	
	\noindent Examining the resulting VTC, we find that the output voltage goes to 0, as expected, when both inputs are 1. Additionally, we find that $V_m = 0.8257\ \text{V}$, which is exactly the same value we found in our netlist.
	
	\begin{figure}[H]
		\centerline{\includegraphics[width=0.8\textwidth]{nand_vtc_schem.png}}
		\caption{VTC from Schematic Simulation}
		\label{fig::nand_vtc_schem}
	\end{figure}	
	
	For completeness, we examine the VTC for other combinations of inputs that result in an output logical change. We have summarized these results in Tables \ref{table::vtc_netlist} and \ref{table::vtc_schematic}. Comparing the results from both tables, we find that the netlist simulation results are identical to the schematic simulation results, which validates the behavior of our layout. 
	
	\begin{table}[H]
	\begin{center}
	\caption{VTC Netlist Results}
	\label{table::vtc_netlist}
	\begin{tabular}{| c | c | c |}
		\hline
		\texttt{a} & \texttt{b} & \texttt{Vm}\\
		\hline	
		$0 \rightarrow 1$ & $1$ & $0.8257\ \text{V}$\\
		\hline	
		$1$ & $0 \rightarrow 1$ & $0.8195\ \text{V}$\\
		\hline	
		$0 \rightarrow 1$ & $0 \rightarrow 1$ & $0.9108\ \text{V}$\\
		\hline
	\end{tabular}
	\end{center}
	\end{table}
	
	\begin{table}[H]
	\begin{center}
	\caption{VTC Schematic Results}
	\label{table::vtc_schematic}
	\begin{tabular}{| c | c | c |}
		\hline
		\texttt{a} & \texttt{b} & \texttt{Vm}\\
		\hline	
		$0 \rightarrow 1$ & $1$ & $0.8257\ \text{V}$\\
		\hline	
		$1$ & $0 \rightarrow 1$ & $0.8195\ \text{V}$\\
		\hline	
		$0 \rightarrow 1$ & $0 \rightarrow 1$ & $0.9108\ \text{V}$\\
		\hline
	\end{tabular}
	\end{center}
	\end{table}
	
	\subsubsection{Noise Analysis}
	
	Next, we perform noise analysis in both the netlist and schematic simulations and compare the results. For this noise analysis, we are specifically interested in the noise margins, which can be computed as follows:
	
	\begin{equation}
		NM_H = V_{OH} - V_{IH}
	\end{equation}
	
	 \begin{equation}
		NM_L = V_{IL} - V_{OL}
	\end{equation}
	
	\noindent $V_{IH}$ and $V_{IL}$ are the unity points of the gain function. We specifically use the SPICE circuit shown in Figure \ref{fig::nand_noise_analysis} to measure the unity gain points in our netlist. These points then allow us to solve for the noise margins.	
	
	\begin{figure}[H]
		\lstinputlisting[style=nocoloring,frame=single,basicstyle=\fontsize{7}{7}\selectfont\ttfamily]{./src/nand_noise_analysis.spice}
		\caption{SPICE Test Circuit to Perform Noise Analysis on Netlist}
		\label{fig::nand_noise_analysis_test_circuit}
	\end{figure}
	
	Examining the netlist, we see that input \texttt{A} is connected to a voltage source swept from 0V to 1.8V in 1mV steps and input \texttt{B} is connected to \texttt{VDD}. Note that these input connections are identical to what was used in Figure \ref{fig::nand_vtc_test_circuit}. Our noise analysis results captured with this test circuit are shown in Figure \ref{fig::nand_noise_analysis_schem}.
	
	\begin{figure}[H]
		\centerline{\includegraphics[width=0.8\textwidth]{nand_noise_analysis.png}}
		\caption{Noise Analysis Results from Netlist Simulation}
		\label{fig::nand_noise_analysis}
	\end{figure}
	
	\noindent Examining the terminal outputs, we see that $V_{IL} = 0.6926\ \text{V}$ and $V_{IH} = 0.9264\ \text{V}$. This implies that $NM_H = 1.8\ \text{V} - 0.9264\ \text{V} = 0.8736\ \text{V}$ and $NM_L = 0.6926\ \text{V} - 0\ \text{V} = 0.6926\ \text{V}$. Using the test circuit shown in Figure \ref{fig::nand_vtc_schem_test_circuit}, we can perform similar analysis for our schematic simulations. These results are displayed in Figure \ref{fig::nand_noise_analysis_schem}.
	
	\begin{figure}[H]
		\centerline{\includegraphics[width=0.8\textwidth]{nand_noise_analysis_schem.png}}
		\caption{Noise Analysis Results from Schematic Simulation}
		\label{fig::nand_noise_analysis_schem}
	\end{figure}
	
	\noindent Examining the schematic simulation outputs, we find that $V_{IL} = 0.6926\ \text{V}$ and $V_{IH} = 0.9264\ \text{V}$, identical to what we found in the netlist simulation. This implies that the noise margins are also identical ($NM_H = 0.8736\ \text{V}$ and $NM_L = 0.6926\ \text{V}$).	
	
	For completeness, we also examine the noise margins for other combinations of inputs that result in an output logical change. We have summarized these results in Tables \ref{table::nand_gate_noise_analysis} and \ref{table::nand_gate_noise_analysis_schem}. Comparing the results from both tables, we find that the netlist simulation results are identical to the schematic simulation results, which validates the behavior of our layout.
	
	\begin{table}[H]
	\begin{center}
	\caption{Noise Margins from Netlist Simulation}
	\label{table::nand_gate_noise_analysis}
	\begin{tabular}{| c | c | c | c | c | c |}
		\hline
		\texttt{a} & \texttt{b} & \texttt{Vih} & \texttt{Vil} & \texttt{Nmh} & \texttt{Nml} \\
		\hline	
		$0 \rightarrow 1$ & $1$ & $0.9264 \text{V}$ & $0.6926 \text{V}$ & $0.8736 \text{V}$ & $0.6926 \text{V}$\\
		\hline	
		$1$ & $0 \rightarrow 1$ & $0.9184 \text{V}$ & $0.7116 \text{V}$ & $0.8816 \text{V}$ & $0.7116 \text{V}$\\
		\hline	
		$0 \rightarrow 1$ & $0 \rightarrow 1$ & $1.0224 \text{V}$ & $0.8046 \text{V}$ & $0.7776 \text{V}$ & $0.7776 \text{V}$\\
		\hline
	\end{tabular}
	\end{center}
	\end{table}
	
	\begin{table}[H]
	\begin{center}
	\caption{Noise Margins from Schematic Simulation}
	\label{table::nand_gate_noise_analysis_schem}
	\begin{tabular}{| c | c | c | c | c | c |}
		\hline
		\texttt{a} & \texttt{b} & \texttt{Vih} & \texttt{Vil} & \texttt{Nmh} & \texttt{Nml} \\
		\hline	
		$0 \rightarrow 1$ & $1$ & $0.9264 \text{V}$ & $0.6926 \text{V}$ & $0.8736 \text{V}$ & $0.6926 \text{V}$\\
		\hline	
		$1$ & $0 \rightarrow 1$ & $0.9184 \text{V}$ & $0.7116 \text{V}$ & $0.8816 \text{V}$ & $0.7116 \text{V}$\\
		\hline	
		$0 \rightarrow 1$ & $0 \rightarrow 1$ & $1.0224 \text{V}$ & $0.8046 \text{V}$ & $0.7776 \text{V}$ & $0.7776 \text{V}$\\
		\hline
	\end{tabular}
	\end{center}
	\end{table}
	
	\subsubsection{Delay Analysis}
	
	
	Then, we perform delay analysis for the netlist and schematic simulations and compare the results. The SPICE circuit we use to test the netlist is shown in Figure \ref{fig::nand_delay_analysis_test_circuit}.
	
	\begin{figure}[H]	\lstinputlisting[style=nocoloring,frame=single,basicstyle=\fontsize{7}{7}\selectfont\ttfamily]{./src/nand_delay_analysis.spice}
		\caption{SPICE Test Circuit to Perform Delay Analysis on Netlist}
		\label{fig::nand_delay_analysis_test_circuit}
	\end{figure}
	
	\noindent Compared to Figures \ref{fig::nand_vtc_test_circuit} and \ref{fig::nand_noise_analysis_test_circuit}, we see that our input \texttt{A} has been replaced with a pulsed voltage source. We can use rising edges of the input to measure the low-to-high propagation delay ($t_{plh}$) and falling edges of the input to measure high-to-low propagation delay ($t_{phl}$). The measured values for the netlist simulation are shown in Figure \ref{fig::nand_delay_analysis}.
	
	\begin{figure}[H]
		\centerline{\includegraphics[width=0.8\textwidth]{nand_delay_analysis.png}}
		\caption{Delay Analysis Results from Netlist Simulation}
		\label{fig::nand_delay_analysis}
	\end{figure}
	
	\noindent Examining the terminal output, we find that the propagation delays are $t_{phl} = 20.366\ \text{ps}$ and $t_{plh} = 33.469\ \text{ps}$. We can perform similar analysis for a schematic simulation using the circuit shown in Figure \ref{fig::nand_delay_analysis_test_circuit_schem}.
	
	\begin{figure}[H]
		\centerline{\includegraphics[width=0.8\textwidth]{nand_delay_analysis_test_circuit.png}}
		\caption{Schematic Test Circuit for NAND Gate Delay Analysis}
		\label{fig::nand_delay_analysis_test_circuit_schem}
	\end{figure}
	
	\noindent Using the test circuit, we measure the propagation delays, which are shown in Figure \ref{fig::nand_delay_analysis_schem}.
	
	\begin{figure}[H]
		\centerline{\includegraphics[width=0.8\textwidth]{nand_delay_analysis_schem.png}}
		\caption{Delay Analysis Results from Schematic Simulation}
		\label{fig::nand_delay_analysis_schem}
	\end{figure}
	
	\noindent Examining the figure output, we find that $t_{phl} = 20.859\ \text{ps}$ and $t_{plh} = 33.178\ \text{ps}$. These results are approximately the same as our netlist simulation, and the slight differences can be explained by minor differences in the layout-based capacitances. However, overall, these findings indicate correct performance.
	
	For completeness, we also measure the propagation delays for other combinations of inputs. These results are summarized in Table \ref{table::nand_gate_delay_analysis} and \ref{table::nand_gate_delay_analysis_schem}. Comparing the results, we see that the netlist and schematic propagation delays are approximately the same for other combinations of inputs.
	 
	\begin{table}[H]
	\begin{center}
	\caption{Delay from Netlist Simulation}
	\label{table::nand_gate_delay_analysis}
	\begin{tabular}{| c | c | c || c | c | c |}
		\hline
		\texttt{a} & \texttt{b} & \texttt{tphl} & \texttt{a} & \texttt{b} & \texttt{tplh} \\
		\hline	
		$0 \rightarrow 1$ & $1$ & $20.336\ \text{ps}$ & $1 \rightarrow 0$ & $1$ & $33.469\ \text{ps}$\\
		\hline	
		$1$ & $0 \rightarrow 1$ & $27.766\ \text{ps}$ & $1$ & $1 \rightarrow 0$ & $47.659\ \text{ps}$\\
		\hline	
		$0 \rightarrow 1$ & $0 \rightarrow 1$ & $29.263\ \text{ps}$ & $1 \rightarrow 0$ & $1 \rightarrow 0$ & $26.400\ \text{ps}$\\
		\hline
	\end{tabular}
	\end{center}
	\end{table}
	
	\begin{table}[H]
	\begin{center}
	\caption{Delay from Schematic Simulation}
	\label{table::nand_gate_delay_analysis_schem}
	\begin{tabular}{| c | c | c || c | c | c |}
		\hline
		\texttt{a} & \texttt{b} & \texttt{tphl} & \texttt{a} & \texttt{b} & \texttt{tplh} \\
		\hline	
		$0 \rightarrow 1$ & $1$ & $20.859\ \text{ps}$ & $1 \rightarrow 0$ & $1$ & $33.178\ \text{ps}$\\
		\hline	
		$1$ & $0 \rightarrow 1$ & $28.416\ \text{ps}$ & $1$ & $1 \rightarrow 0$ & $48.137\ \text{ps}$\\
		\hline	
		$0 \rightarrow 1$ & $0 \rightarrow 1$ & $29.297\ \text{ps}$ & $1 \rightarrow 0$ & $1 \rightarrow 0$ & $26.097\ \text{ps}$\\
		\hline
	\end{tabular}
	\end{center}
	\end{table}
	
	\subsubsection{Power Analysis}
	
	Finally, we examine the power consumption of our layout in netlist-based simulation and compare the results to our schematic-based simulation. Our SPICE test circuit to measure the power consumption of our netlist is included in Figure \ref{fig::nand_power_analysis_test_circuit}.
	
	\begin{figure}[H]
		\lstinputlisting[style=nocoloring,frame=single,basicstyle=\fontsize{7}{7}\selectfont\ttfamily]{./src/nand_power_analysis.spice}
		\caption{SPICE Test Circuit to Perform Power Analysis on Netlist}
		\label{fig::nand_power_analysis_test_circuit}
	\end{figure}
	
	\noindent Similar to Figure \ref{fig::nand_delay_analysis_test_circuit}, we find that the input \texttt{A} is being driven by a pulsed input source, while the input \texttt{B} is tied to \texttt{VDD}. The pulsed input source has a slightly larger period, which results in a slightly longer simulation time. We have also included a 4fF parasitic capacitance to match the work we did in Lab 1. The resulting power consumption is included in Figure \ref{fig::nand_power_analysis}.
	
	\begin{figure}[H]
		\centerline{\includegraphics[width=0.8\textwidth]{nand_power_analysis.png}}
		\caption{Power Analysis Results from Netlist Simulation}
		\label{fig::nand_power_analysis}
	\end{figure}
	
	\noindent Examining the outputs of Figure \ref{fig::nand_power_analysis}, we find that the NAND gate power consumption is $1.34724\ \mu{W}$. We can compare this power consumption to the power consumption of the NAND gate in our schematic simulation. To measure the power consumption of our schematic we use the test circuit shown in Figure \ref{fig::nand_power_analysis_test_circuit_schem}.
	
	\begin{figure}[H]
		\centerline{\includegraphics[width=0.8\textwidth]{nand_power_analysis_test_circuit.png}}
		\caption{Schematic Test Circuit for NAND Gate Power Analysis}
		\label{fig::nand_power_analysis_test_circuit_schem}
	\end{figure}
	
	\noindent Our simulation results are included in Figure \ref{fig::nand_power_analysis_schem}.
	
	\begin{figure}[H]
		\centerline{\includegraphics[width=0.8\textwidth]{nand_power_analysis_schem.png}}
		\caption{Power Analysis Results from Schematic Simulation}
		\label{fig::nand_power_analysis_schem}
	\end{figure}
	
	\noindent Examining the terminal outputs, we find that the power consumption is $1.34880\ \mu{W}$. This is almost identical to what we find in the netlist simulation. Any slight differences can be explained by the presence of small layout-based capacitances.
	
	For completeness, we also measure the propagation delays for other combinations of inputs. The power consumption for the netlist-based simulation and the schematic simulation are summarized in Tables \ref{table::nand_gate_power_analysis} and \ref{table::nand_gate_power_analysis_schem}. Because the power consumption of the schematic is approximately the same as the netlist, we can conclude that our layout is valid.
	
	\begin{table}[H]
	\begin{center}
	\caption{Power Consumption from Netlist Simulation}
	\label{table::nand_gate_power_analysis}
	\begin{tabular}{| c | c | c |}
		\hline
		\texttt{a} & \texttt{b} & \texttt{Power}\\
		\hline	
		$0 \rightarrow 1 \rightarrow 0$ & $1$ & $1.34724{\mu}W$ \\
		\hline	
		$1$ & $0 \rightarrow 1 \rightarrow 0$ & $1.85866{\mu}W$ \\
		\hline	
		$0 \rightarrow 1 \rightarrow 0$ & $0 \rightarrow 1 \rightarrow 0$ & $1.44641{\mu}W$\\
		\hline
	\end{tabular}
	\end{center}
	\end{table}
	
	\begin{table}[H]
	\begin{center}
	\caption{Power Consumption from Schematic Simulation}
	\label{table::nand_gate_power_analysis_schem}
	\begin{tabular}{| c | c | c |}
		\hline
		\texttt{a} & \texttt{b} & \texttt{Power}\\
		\hline	
		$0 \rightarrow 1 \rightarrow 0$ & $1$ & $1.34880{\mu}W$ \\
		\hline	
		$1$ & $0 \rightarrow 1 \rightarrow 0$ & $1.88943{\mu}W$ \\
		\hline	
		$0 \rightarrow 1 \rightarrow 0$ & $0 \rightarrow 1 \rightarrow 0$ & $1.44446{\mu}W$\\
		\hline
	\end{tabular}
	\end{center}
	\end{table}
	
	\section{NOR Gates}
	
	In this section, we create the layout of a NOR gate and examine its contents. Then, we validate our layout by performing DRC. After our design is validated, we extract a SPICE netlist. We also create a comparable schematic. Finally, we compare our layout-based netlists to the schematic in simulation. In these simulations, we specifically analyze the voltage transfer characteristics (VTC), noise margins, propagation delays, and power consumption of our gate. Comparable results confirm the correctness of our layout.
	
	\subsection{Layout}
	
	In this section, we review our NOR gate layout, which is displayed in Figure \ref{fig::nor_layout}. The NAND gate is composed of two NMOS transistors in parallel and two PMOS transistors in series. The NMOS transistors lie along the n-diffusion region and the PMOS transistor lie along the p-diffusion region. The n-diffusion region is placed in the p-type body, and the p-diffusion region is placed in an n-well. Polysilicon is used for the transistor gates. The layout also includes metal1 wires for ground and VDD. The ground wire is connected to the source of both NMOS transistors with local interconnect (li) material, and the VDD wire is connected to the source of leftmost PMOS transistor with li material. The drain of the rightmost PMOS transistor is connected to the drain of both NMOS transistors with li material to form the gate output. Metal1 contacts are placed on the gate inputs and outputs to allow for longer-distance routing. Depending on the distance that the I/O must be routed, li material may be sufficient for the connections. To prevent latchup, the ground wire is also connected to p-substrate diffusion with a substrate tap, and the VDD wire is connected to n-substrate diffusion with a well tap.
	
	\begin{figure}[H]
		\centerline{\includegraphics[width=0.3\textwidth]{nor_layout.png}}
		\caption{NOR Gate Layout}
		\label{fig::nor_layout}
	\end{figure}
	
	Most of the transistor properties are defined by the size of the NMOS and PMOS transistor channels, whose dimensions are shown in Figures \ref{fig::nor_nmos_channel_sizing} and \ref{fig::nor_pmos_channel_sizing}. Note that we have designed the transistors with the minimum channel length for the sky130A technology (150 nm). The channel widths have been chosen to match the delay of an inverter with $1 {\mu}m$ NMOS channel widths and $2 {\mu}m$ PMOS channel widths. With respect to the inverter, the NOR gate PMOS channel widths have been doubled to $4 {\mu}m$ to maintain the same equivalent resistance. The NOR gate NMOS channel widths are kept the same to maintain the same equivalent resistance for the slowest case (when only when transistor is on).
	
	\begin{figure}[H]
		\centerline{\includegraphics[width=0.5\textwidth]{nor_nmos_channel_sizing.png}}
		\caption{Checking NMOS Channel Size}
		\label{fig::nor_nmos_channel_sizing}
	\end{figure}
	
	\begin{figure}[H]
		\centerline{\includegraphics[width=0.5\textwidth]{nor_pmos_channel_sizing.png}}
		\caption{Checking PMOS Channel Size}
		\label{fig::nor_pmos_channel_sizing}
	\end{figure}
	
	For our layout to be valid, it must pass design rule checks (DRC). In Figures \ref{fig::nor_drc_errors_terminal} and \ref{fig::nor_drc_errors_drcmgr}, we confirm that our design meets DRC in two different ways. First, in Figure \ref{fig::nor_drc_errors_terminal}, we use terminal commands to perform a DRC. The \texttt{select} command, selects the entire design, and the \texttt{drc count} and \texttt{drc why} commands return the count and reason for DRC errors respectively. Examining the command outputs, we see that there are no DRC errors. Next, in Figure \ref{fig::nor_drc_errors_drcmgr}, we use the drcmgr to check for DRC errors. Examining the drcmgr output, we see that there are no DRC errors. The primary GUI toolbar also displays a DRC count of 0, which further confirms the validity of our layout.
	
	\begin{figure}[H]
		\centerline{\includegraphics[width=0.5\textwidth]{nor_drc_errors_terminal.png}}
		\caption{Checking DRC Errors from Terminal}
		\label{fig::nor_drc_errors_terminal}
	\end{figure}
	
	\begin{figure}[H]
		\centerline{\includegraphics[width=0.5\textwidth]{nor_drc_errors_drcmgr.png}}
		\caption{Checking DRC Errors with drcmgr}
		\label{fig::nor_drc_errors_drcmgr}
	\end{figure}
	
	\subsection{Netlist}
	
	\begin{figure}[H]
		\lstinputlisting[style=nocoloring,frame=single]{./src/nor.spice}
		\caption{Extracted SPICE Netlist for NOR Gate}
		\label{fig::nor_netlist}
	\end{figure}
	
	\subsection{Schematic Simulation}
	
	\begin{figure}[H]
		\centerline{\includegraphics[width=0.5\textwidth]{nor_schematic.png}}
		\caption{NOR Schematic}
		\label{fig::nor_sechematic}
	\end{figure}
	
	\begin{figure}[H]
		\centerline{\includegraphics[width=0.3\textwidth]{nor_symbol.png}}
		\caption{NOR Schematic Symbol}
		\label{fig::nor_symbol}
	\end{figure}
	
	\subsection{Simulation Results}
	
	\subsubsection{VTC}
	
	\begin{figure}[H]
		\lstinputlisting[style=nocoloring,frame=single,basicstyle=\fontsize{7}{7}\selectfont\ttfamily]{./src/nor_vtc.spice}
		\caption{SPICE Test Circuit to Extract VTC from Netlist}
		\label{fig::nor_vtc_test_circuit}
	\end{figure}
	
	\begin{figure}[H]
		\centerline{\includegraphics[width=0.8\textwidth]{nor_vtc.png}}
		\caption{VTC from Netlist Simulation}
		\label{fig::nor_vtc}
	\end{figure}
	
	\begin{figure}[H]
		\centerline{\includegraphics[width=0.8\textwidth]{nor_vtc_test_circuit.png}}
		\caption{Schematic Test Circuit for Collect NOR Gate VTC}
		\label{fig::nor_vtc_schem_test_circuit}
	\end{figure}
	
	\begin{figure}[H]
		\centerline{\includegraphics[width=0.8\textwidth]{nor_vtc_schem.png}}
		\caption{VTC from Schematic Simulation}
		\label{fig::nor_vtc_schem}
	\end{figure}
	
	\begin{table}[H]
	\begin{center}
	\caption{VTC Netlist Results}
	\label{table::nor_vtc_netlist}
	\begin{tabular}{| c | c | c |}
		\hline
		\texttt{a} & \texttt{b} & \texttt{Vm}\\
		\hline	
		$0 \rightarrow 1$ & $0$ & $0.8952\ \text{V}$\\
		\hline	
		$0$ & $0 \rightarrow 1$ & $0.8919\ \text{V}$\\
		\hline	
		$0 \rightarrow 1$ & $0 \rightarrow 1$ & $0.8078\ \text{V}$\\
		\hline
	\end{tabular}
	\end{center}
	\end{table}
	
	\begin{table}[H]
	\begin{center}
	\caption{VTC Schematic Results}
	\label{table::nor_vtc_schematic}
	\begin{tabular}{| c | c | c |}
		\hline
		\texttt{a} & \texttt{b} & \texttt{Vm}\\
		\hline	
		$0 \rightarrow 1$ & $0$ & $0.8952\ \text{V}$\\
		\hline	
		$0$ & $0 \rightarrow 1$ & $0.8919\ \text{V}$\\
		\hline	
		$0 \rightarrow 1$ & $0 \rightarrow 1$ & $0.8078\ \text{V}$\\
		\hline
	\end{tabular}
	\end{center}
	\end{table}
	
	\subsubsection{Noise Analysis}
	
	\begin{figure}[H]
		\lstinputlisting[style=nocoloring,frame=single,basicstyle=\fontsize{7}{7}\selectfont\ttfamily]{./src/nor_noise_analysis.spice}
		\caption{SPICE Test Circuit to Perform Noise Analysis on Netlist}
		\label{fig::nor_noise_analysis_test_circuit}
	\end{figure}
	
	\begin{figure}[H]
		\centerline{\includegraphics[width=0.8\textwidth]{nor_noise_analysis.png}}
		\caption{Noise Analysis Results from Netlist Simulation}
		\label{fig::nor_noise_analysis}
	\end{figure}
	
	\begin{figure}[H]
		\centerline{\includegraphics[width=0.8\textwidth]{nor_noise_analysis_schem.png}}
		\caption{Noise Analysis Results from Schematic Simulation}
		\label{fig::nor_noise_analysis_schem}
	\end{figure}
	
	\begin{table}[H]
	\begin{center}
	\caption{Noise Margins from Netlist Simulation}
	\label{table::nor_gate_noise_analysis}
	\begin{tabular}{| c | c | c | c | c | c |}
		\hline
		\texttt{a} & \texttt{b} & \texttt{Vih} & \texttt{Vil} & \texttt{Nmh} & \texttt{Nml} \\
		\hline	
		$0 \rightarrow 1$ & $0$ & $0.9764 \text{V}$ & $0.7756 \text{V}$ & $0.8326 \text{V}$ & $0.7756 \text{V}$\\
		\hline	
		$0$ & $0 \rightarrow 1$ & $1.0134 \text{V}$ & $0.7706 \text{V}$ & $0.7660 \text{V}$ & $0.7706 \text{V}$\\
		\hline	
		$0 \rightarrow 1$ & $0 \rightarrow 1$ & $0.7006 \text{V}$ & $0.8844 \text{V}$ & $0.9156 \text{V}$ & $0.7006 \text{V}$\\
		\hline
	\end{tabular}
	\end{center}
	\end{table}
	
	\begin{table}[H]
	\begin{center}
	\caption{Noise Margins from Schematic Simulation}
	\label{table::nor_gate_noise_analysis_schem}
	\begin{tabular}{| c | c | c | c | c | c |}
		\hline
		\texttt{a} & \texttt{b} & \texttt{Vih} & \texttt{Vil} & \texttt{Nmh} & \texttt{Nml} \\
		\hline	
		$0 \rightarrow 1$ & $0$ & $0.9764 \text{V}$ & $0.7756 \text{V}$ & $0.8326 \text{V}$ & $0.7756 \text{V}$\\
		\hline	
		$0$ & $0 \rightarrow 1$ & $1.0134 \text{V}$ & $0.7706 \text{V}$ & $0.7660 \text{V}$ & $0.7706 \text{V}$\\
		\hline	
		$0 \rightarrow 1$ & $0 \rightarrow 1$ & $0.7006 \text{V}$ & $0.8844 \text{V}$ & $0.9156 \text{V}$ & $0.7006 \text{V}$\\
		\hline
	\end{tabular}
	\end{center}
	\end{table}
	
	\subsubsection{Delay Analysis}
	\begin{figure}[H]
		\lstinputlisting[style=nocoloring,frame=single,basicstyle=\fontsize{7}{7}\selectfont\ttfamily]{./src/nor_delay_analysis.spice}
		\caption{SPICE Test Circuit to Perform Delay Analysis on Netlist}
		\label{fig::nor_delay_analysis_test_circuit}
	\end{figure}
	
	\begin{figure}[H]
		\centerline{\includegraphics[width=0.8\textwidth]{nor_delay_analysis.png}}
		\caption{Delay Analysis Results from Netlist Simulation}
		\label{fig::nor_delay_analysis}
	\end{figure}
	
	\begin{figure}[H]
		\centerline{\includegraphics[width=0.8\textwidth]{nor_delay_analysis_test_circuit.png}}
		\caption{Schematic Test Circuit for NOR Gate Delay Analysis}
		\label{fig::nor_delay_analysis_test_circuit_schem}
	\end{figure}
	
	\begin{figure}[H]
		\centerline{\includegraphics[width=0.8\textwidth]{nor_delay_analysis_schem.png}}
		\caption{Delay Analysis Result from Schematic Simulation}
		\label{fig::nor_delay_analysis_schem}
	\end{figure}
	
	\begin{table}[H]
	\begin{center}
	\caption{Delay from Netlist Simulation}
	\label{table::nor_gate_delay_analysis}
	\begin{tabular}{| c | c | c || c | c | c |}
		\hline
		\texttt{a} & \texttt{b} & \texttt{tphl} & \texttt{a} & \texttt{b} & \texttt{tplh} \\
		\hline	
		$0 \rightarrow 1$ & $0$ & $45.126\ \text{ps}$ & $1 \rightarrow 0$ & $0$ & $45.536\ \text{ps}$\\
		\hline	
		$0$ & $0 \rightarrow 1$ & $30.061\ \text{ps}$ & $0$ & $1 \rightarrow 0$ & $30.975\ \text{ps}$\\
		\hline	
		$0 \rightarrow 1$ & $0 \rightarrow 1$ & $24.033\ \text{ps}$ & $1 \rightarrow 0$ & $1 \rightarrow 0$ & $43.556\ \text{ps}$\\
		\hline
	\end{tabular}
	\end{center}
	\end{table}
	
	\begin{table}[H]
	\begin{center}
	\caption{Delay from Schematic Simulation}
	\label{table::nor_gate_delay_analysis_schem}
	\begin{tabular}{| c | c | c || c | c | c |}
		\hline
		\texttt{a} & \texttt{b} & \texttt{tphl} & \texttt{a} & \texttt{b} & \texttt{tplh} \\
		\hline	
		$0 \rightarrow 1$ & $0$ & $44.975\ \text{ps}$ & $1 \rightarrow 0$ & $0$ & $45.841\ \text{ps}$\\
		\hline	
		$0$ & $0 \rightarrow 1$ & $29.139\ \text{ps}$ & $0$ & $1 \rightarrow 0$ & $29.942\ \text{ps}$\\
		\hline	
		$0 \rightarrow 1$ & $0 \rightarrow 1$ & $22.437\ \text{ps}$ & $1 \rightarrow 0$ & $1 \rightarrow 0$ & $40.998\ \text{ps}$\\
		\hline
	\end{tabular}
	\end{center}
	\end{table}
	
	\subsubsection{Power Analysis}
	\begin{figure}[H]
		\lstinputlisting[style=nocoloring,frame=single,basicstyle=\fontsize{7}{7}\selectfont\ttfamily]{./src/nor_power_analysis.spice}
		\caption{SPICE Test Circuit to Perform Power Analysis on Netlist}
		\label{fig::nor_power_analysis_test_circuit}
	\end{figure}
	
	\begin{figure}[H]
		\centerline{\includegraphics[width=0.8\textwidth]{nor_power_analysis.png}}
		\caption{Power Analysis Results from Netlist Simulation}
		\label{fig::nor_power_analysis}
	\end{figure}
	
	\begin{figure}[H]
		\centerline{\includegraphics[width=0.8\textwidth]{nor_power_analysis_test_circuit.png}}
		\caption{Schematic Test Circuit for NOR Gate Power Analysis}
		\label{fig::nor_power_analysis_test_circuit_schem}
	\end{figure}
	
	\begin{figure}[H]
		\centerline{\includegraphics[width=0.8\textwidth]{nor_power_analysis_schem.png}}
		\caption{Power Analysis Results from Schematic Simulation}
		\label{fig::nor_power_analysis_schem}
	\end{figure}
	
	\begin{table}[H]
	\begin{center}
	\caption{Power Consumption from Netlist Simulation}
	\label{table::nor_gate_power_analysis}
	\begin{tabular}{| c | c | c |}
		\hline
		\texttt{a} & \texttt{b} & \texttt{Power}\\
		\hline	
		$0 \rightarrow 1 \rightarrow 0$ & $0$ & $2.24763{\mu}W$ \\
		\hline	
		$0$ & $0 \rightarrow 1 \rightarrow 0$ & $1.43645{\mu}W$ \\
		\hline	
		$0 \rightarrow 1 \rightarrow 0$ & $0 \rightarrow 1 \rightarrow 0$ & $1.67940{\mu}W$\\
		\hline
	\end{tabular}
	\end{center}
	\end{table}
	
	\begin{table}[H]
	\begin{center}
	\caption{Power Consumption from Schematic Simulation}
	\label{table::nor_gate_power_analysis_schem}
	\begin{tabular}{| c | c | c |}
		\hline
		\texttt{a} & \texttt{b} & \texttt{Power}\\
		\hline	
		$0 \rightarrow 1 \rightarrow 0$ & $0$ & $2.28022{\mu}W$ \\
		\hline	
		$0$ & $0 \rightarrow 1 \rightarrow 0$ & $1.37674{\mu}W$ \\
		\hline	
		$0 \rightarrow 1 \rightarrow 0$ & $0 \rightarrow 1 \rightarrow 0$ & $1.61932{\mu}W$\\
		\hline
	\end{tabular}
	\end{center}
	\end{table}
	
	\bibliographystyle{IEEEtran}
	\bibliography{IEEEabrv,sources}
\end{document}